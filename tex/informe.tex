\documentclass[a4paper, 10pt, twoside]{article}

\usepackage[top=1in, bottom=1in, left=1in, right=1in]{geometry}
\usepackage[utf8]{inputenc}
\usepackage[spanish, es-ucroman, es-noquoting]{babel}
\usepackage{setspace}
\usepackage{fancyhdr}
\usepackage{lastpage}
\usepackage{amsmath}
\usepackage{amsfonts}
\usepackage{amsthm}
\usepackage{verbatim}
\usepackage{fancyvrb}
\usepackage{graphicx}
\usepackage{float}
\usepackage{enumitem} % Provee macro \setlist
\usepackage{tabularx}
\usepackage{multirow}
\usepackage{hyperref}
\usepackage{xspace}
\usepackage[toc, page]{appendix}


%%%%%%%%%% Constantes - Inicio %%%%%%%%%%
\newcommand{\titulo}{Trabajo Práctico 2}
\newcommand{\materia}{Bases de Datos}
\newcommand{\integrantes}{Delgado · Lovisolo · Petaccio · Rebecchi}
\newcommand{\cuatrimestre}{Segundo Cuatrimestre de 2014}
%%%%%%%%%% Constantes - Fin %%%%%%%%%%


%%%%%%%%%% Configuración de Fancyhdr - Inicio %%%%%%%%%%
\pagestyle{fancy}
\thispagestyle{fancy}
\lhead{\titulo\ · \materia}
\rhead{\integrantes}
\renewcommand{\footrulewidth}{0.4pt}
\cfoot{\thepage /\pageref{LastPage}}

\fancypagestyle{caratula} {
   \fancyhf{}
   \cfoot{\thepage /\pageref{LastPage}}
   \renewcommand{\headrulewidth}{0pt}
   \renewcommand{\footrulewidth}{0pt}
}
%%%%%%%%%% Configuración de Fancyhdr - Fin %%%%%%%%%%


%%%%%%%%%% Insertar gráfico - Inicio %%%%%%%%%%
\newcommand{\grafico}[3]{
  \begin{figure}[H]
    \includegraphics[type=pdf,ext=.pdf,read=.pdf]{#1}
    \caption{#2}
    \label{#3}
  \end{figure}
}
%%%%%%%%%% Insertar gráfico - Fin %%%%%%%%%%


%%%%%%%%%% Miscelánea - Inicio %%%%%%%%%%
% Evita que el documento se estire verticalmente para ocupar el espacio vacío
% en cada página.
\raggedbottom

% Separación entre párrafos.
\setlength{\parskip}{0.5em}

% Separación entre elementos de listas.
\setlist{itemsep=0.5em}

% Asigna la traducción de la palabra 'Appendices'.
\renewcommand{\appendixtocname}{Apéndices}
\renewcommand{\appendixpagename}{Apéndices}
%%%%%%%%%% Miscelánea - Fin %%%%%%%%%%


\begin{document}


%%%%%%%%%%%%%%%%%%%%%%%%%%%%%%%%%%%%%%%%%%%%%%%%%%%%%%%%%%%%%%%%%%%%%%%%%%%%%%%%
%% Carátula                                                                   %%
%%%%%%%%%%%%%%%%%%%%%%%%%%%%%%%%%%%%%%%%%%%%%%%%%%%%%%%%%%%%%%%%%%%%%%%%%%%%%%%%


\thispagestyle{caratula}

\begin{center}

\includegraphics[height=2cm]{DC.png} 
\hfill
\includegraphics[height=2cm]{UBA.jpg} 

\vspace{2cm}

Departamento de Computación,\\
Facultad de Ciencias Exactas y Naturales,\\
Universidad de Buenos Aires

\vspace{4cm}

\begin{Huge}
\titulo
\end{Huge}

\vspace{0.5cm}

\begin{Large}
\materia
\end{Large}

\vspace{1cm}

\cuatrimestre

\vspace{4cm}

\begin{tabular}{|c|c|c|}
\hline
Apellido y Nombre & LU & E-mail\\
\hline
Delgado, Alejandro N.  & 601/11 & nahueldelgado@gmail.com\\
Lovisolo, Leandro      & 645/11 & leandro@leandro.me\\
Petaccio, Lautaro José & 443/11 & lausuper@gmail.com\\
Rebecchi, Alejandro    & 15/10  & alejandrorebecchi@gmail.com\\
\hline
\end{tabular}

\end{center}

\newpage


%%%%%%%%%%%%%%%%%%%%%%%%%%%%%%%%%%%%%%%%%%%%%%%%%%%%%%%%%%%%%%%%%%%%%%%%%%%%%%%%
%% Introducción                                                               %%
%%%%%%%%%%%%%%%%%%%%%%%%%%%%%%%%%%%%%%%%%%%%%%%%%%%%%%%%%%%%%%%%%%%%%%%%%%%%%%%%


\section{Introducción}

Pendiente.


%%%%%%%%%%%%%%%%%%%%%%%%%%%%%%%%%%%%%%%%%%%%%%%%%%%%%%%%%%%%%%%%%%%%%%%%%%%%%%%%
%% Estimadores                                                                %%
%%%%%%%%%%%%%%%%%%%%%%%%%%%%%%%%%%%%%%%%%%%%%%%%%%%%%%%%%%%%%%%%%%%%%%%%%%%%%%%%


\section{Estimadores}

\subsection{Classic Histogram}

Este estimador se basa en un modelo de histograma en el cual cada barra representa un rango
de valores del mismo tamaño.
Es decir, al inicializar el estimador, se le pasa como parámetro el numero de barras que se
desea que tenga el histograma
y luego se utiliza dicho valor para dividir el rango de valores en partes iguales.
Para calcular la selectividad por igualdad se asume que todos los valores que pertenecen a
la misma barra tienen la
 misma cantidad de registros, por lo que se divide el numero de registros de dicha barra
  por la cantidad total y el numero de elementos por barra.
% no dividimos a su vez por el ancho de la barra ? SI %
Siguiendo la misma idea, para calcular la selectividad por mayor se suman todos los
registros que corresponden a las barras superiores a la que pertenece el elemento, mas los que se estima que son mayores dentro de la misma barra. Esto se calcula en base a la diferencia con el limite superior del intervalo que lo contiene.

\subsection{Distribution Steps}

En este caso se busca obtener un histograma con barrasde la misma altura variando el
tamaño de los intervalos que representa cada barra. De esta forma se busca tener mayor
 granularidad al en el caso de que para algunos valores tengamos mucha cantidad de
 registros y para otros no. para esto se arma un histograma dividiendo la cantidad de
 registros de la tabla N grupos de igual tamaño, siendo N pasado por parámetro al
 inicializarlo. Además se quiere que la suma de las estimaciones de las selectividades
 por moyor, menos e igual de un mismo valor sumen 1.
Por lo tanto para calcular la estimación de elementos mayores se asume que 1/3 de los
elementos del intervalo son mayores si el valor que se evalúa no es límite de un
intervalo. Para estimar la selectividad de la igualdad, se asume que aproximadamente
 1/3 de los elementos pertenecientes al grupo, son iguales al valor buscado,
 por lo que al sumarse con la selectividad de menor y mayor se llega a 1.

\subsection{Estimador Grupo}

En este estimador se busca aprovechar el tipo de histograma que se arma con distribution steps,
pero se busca mejorar el estimador por igualdad, ya que con una cantidad baja de steps, quedan muchos registros por
grupo y generalmente estimar que 1/3 de ellos son iguales al valor que buscamos trae un error muy grande.
Por lo tanto para estimar la cantidad de elementos iguales a uno dado se divide la cantidad de elementos del step
(siempre es la misma) por el rango de valores del step y luego ésto por la cantidad total de elementos de la base.
El estimador por mayor valor funciona de forma similar al de distribution steps.

%%%%%%%%%%%%%%%%%%%%%%%%%%%%%%%%%%%%%%%%%%%%%%%%%%%%%%%%%%%%%%%%%%%%%%%%%%%%%%%%
%% Análisis Teório                                                            %%
%%%%%%%%%%%%%%%%%%%%%%%%%%%%%%%%%%%%%%%%%%%%%%%%%%%%%%%%%%%%%%%%%%%%%%%%%%%%%%%%


\section{Análisis Teórico}

\subsection{Dataset con distribucion normal}
El estimador Classic Histogram presenta una baja pérfomance bajo una distribución normal, en especial, a la hora de realizar operaciones de mayor o menor.
Esto sucede debdio a que, al ser la altura de los bloques del histograma variables, cualquier búsqueda por mayor o menor que caiga en un bloque
donde la altura es importante, tendrá un error alto, ya que, el cálculo de la selectividad se realiza tomando el punto medio entre la selectividad que dará
el intervalo siguiente y el anterior, haciendo que, en el caso de tener una cantidad importante de tuplas, la división a la mitad del intervalo, dará como máximo,
un error de 0.5 en la selectividad.

Al contrario de Classic Histogram, Distribution Steps presentará una mejor pérformance ante esta distribución. Esto se debe a que no sufre del problema
de los bloques de gran altura. Esto ocurre porque el estimador construye bloques o \textit{steps} de misma altura o cantidad de tuplas, haciendo que el problema 
antes descripto para Classic Histogram no ocurra resultando en un error de estimación de la selectividad menor.

\subsection{Dataset con distribución uniforme}
El estimador Classic Histogram, a la hora de construir sus \textit{barras} o bloques, como la distribución es uniforme en el rango, tendrá \textit{barras}
de igual altura, eliminando la ocurrencia del problema mencionado en el Dataset con distribución normal, ya que ahora, al tener la misma altura, se minimiza
el punto medio de donde se sacará la selectividad, haciendo que el error reduzca considerablemente.

En esta distribución, ambos estimadores tendrán sus bloques de igual tamaño, por lo que el problema que sufría anteriormente Classic Histogram no está
presente para mostrar una diferencia significativa entre estimadores.


%%%%%%%%%%%%%%%%%%%%%%%%%%%%%%%%%%%%%%%%%%%%%%%%%%%%%%%%%%%%%%%%%%%%%%%%%%%%%%%%
%% Análisis Empírico                                                          %%
%%%%%%%%%%%%%%%%%%%%%%%%%%%%%%%%%%%%%%%%%%%%%%%%%%%%%%%%%%%%%%%%%%%%%%%%%%%%%%%%


\section{Análisis Empírico}

\subsection{Experimentación}
Para los resultados generados, se utilizaron los estimadores Classic Histogram, Distribution Steps y el Estimador Grupo 
sobre la distribución normal y la distribución uniforme de las figuras \ref{custom-dataset-uniform} y 
\ref{custom-dataset-uniform} respectivamente, generadas aleatoriamente. 

Se utilizaron queries 50 queries para la evaluación de la selectividad en intervalos de 20 (\textbf{Sel(}=0\textbf{)}, \textbf{Sel(}=19\textbf{)}, ...) y se corrieron tanto para la selectividad por igual como para mayor, con valores del 
parámetro de los estimadores entre 5 y 100 en intervalos de 5.



\subsection{Datasets utilizados}

A continuación se ilustran los datasets utilizados durante el análisis de performance de los estimadores estudiados.

\grafico{custom-dataset-uniform}
        {Distribución uniforme}
        {custom-dataset-uniform}

\grafico{custom-dataset-normal}
        {Distribución normal}
        {custom-dataset-normal}


\subsection{Estimador \texttt{ClassicHistogram}}

\subsubsection{Distribución uniforme}

\grafico{plot-hist-uniform-equal}
        {Distribución uniforme, consulta por igualdad.}
        {plot-hist-uniform-equal}
\grafico{plot-hist-uniform-greater}
        {Distribución uniforme, consulta por mayor.}
        {plot-hist-uniform-greater}

En las figuras \ref{plot-custom-uniform-greater} y \ref{plot-hist-uniform-equal} se pueden observar valores muy bajos
en la pérfomance, acompañados de una gran varianza. En ambas figuras, se puede ver como al iniciar con un valor en parámetro bajo,
la pérformance es más alta, como es de esperarse, ya que, aumentar este valor debería producir una reducción de la pérformance al obtener más precisión.
Atribuímos la gran varianza y anomalías aparentes como pequeñas montañas en los gráficos como pueden verse en los
valores del parámetro 35 y 70 de la figura \ref{plot-hist-uniform-equal} por ejemplo, a errores de punto flotante en el cálculo de la selectividad 
y en el armado del estimador.
El método de predicción de selectividad elejido para la igualdad genera una pérformance muy baja, obteniéndose resultados satisfactorios y similares para
todos los valores probados del parámetro.

\subsubsection{Distribución normal}


\grafico{plot-hist-normal-equal}
        {Distribución normal, consulta por igualdad.}
        {plot-hist-normal-equal}
\grafico{plot-hist-normal-greater}
        {Distribución normal, consulta por mayor.}
        {plot-hist-normal-greater}

Las figuras \ref{plot-hist-normal-equal} y \ref{plot-hist-normal-greater} de selectividad por igualdad y por mayor respectivamente, evidencian la diferencia 
de pérfomance que implica analizar la distribución normal con este estimador.

Como se vió en el análisis teórico anteriormente, la búsqueda de selectividad por mayor, que podía obtener un error máximo 
alto en la selectividad en esta distribución, en especial con valores del parámetro bajo, muestra en la figura \ref{plot-hist-normal-greater} una pérfomance alta 
que decrece contínuamente a medida que se incrementa el valor del parámetro, como era de esperarse, debido a que intervalos más chicos implican 
mayor precisión en la selectividad por mayor.

Por el lado de la igualdad, se muestra una pérfomance alta con un varianza alta con valores bajos del parámetro a comparación de la figura \ref{plot-hist-uniform-equal} 
donde se utiliza una distribución uniforme. Sin embargo, con sucesivos incrementos en el parámetro, el aumento se precisión evidencia una mejora notable.   

\subsection{Estimador \texttt{DistributionSteps}}

\grafico{plot-diststep-uniform-equal}
        {Distribución uniforme, consulta por igualdad.}
        {plot-diststep-uniform-equal}
\grafico{plot-diststep-uniform-greater}
        {Distribución uniforme, consulta por mayor.}
        {plot-diststep-uniform-greater}
\grafico{plot-diststep-normal-equal}
        {Distribución normal, consulta por igualdad.}
        {plot-diststep-normal-equal}
\grafico{plot-diststep-normal-greater}
        {Distribución normal, consulta por mayor.}
        {plot-diststep-normal-greater}


\subsection{Estimador \texttt{EstimadorGrupo}}

\grafico{plot-custom-uniform-equal}
        {Distribución uniforme, consulta por igualdad.}
        {plot-custom-uniform-equal}
\grafico{plot-custom-uniform-greater}
        {Distribución uniforme, consulta por mayor.}
        {plot-custom-uniform-greater}
\grafico{plot-custom-normal-equal}
        {Distribución normal, consulta por igualdad.}
        {plot-custom-normal-equal}
\grafico{plot-custom-normal-greater}
        {Distribución normal, consulta por mayor.}
        {plot-custom-normal-greater}


\subsection{Datasets provistos por la cátedra}

A continuación ilustramos los datasets correspondientes a cada columna de la tabla en la base de datos provista por la cátedra.

\grafico{dataset-c0}
        {Columna \emph{c0}}
        {dataset-columna-c0}

\grafico{dataset-c1}
        {Columna \emph{c1}}
        {dataset-columna-c1}

\grafico{dataset-c2}
        {Columna \emph{c2}}
        {dataset-columna-c2}

\grafico{dataset-c3}
        {Columna \emph{c3}}
        {dataset-columna-c3}

\grafico{dataset-c4}
        {Columna \emph{c4}}
        {dataset-columna-c4}

\grafico{dataset-c5}
        {Columna \emph{c5}}
        {dataset-columna-c5}

\grafico{dataset-c6}
        {Columna \emph{c6}}
        {dataset-columna-c6}

\grafico{dataset-c7}
        {Columna \emph{c7}}
        {dataset-columna-c7}

\grafico{dataset-c8}
        {Columna \emph{c8}}
        {dataset-columna-c8}

\grafico{dataset-c9}
        {Columna \emph{c9}}
        {dataset-columna-c9}


%%%%%%%%%%%%%%%%%%%%%%%%%%%%%%%%%%%%%%%%%%%%%%%%%%%%%%%%%%%%%%%%%%%%%%%%%%%%%%%%
%% Discusión                                                                  %%
%%%%%%%%%%%%%%%%%%%%%%%%%%%%%%%%%%%%%%%%%%%%%%%%%%%%%%%%%%%%%%%%%%%%%%%%%%%%%%%%


\section{Discusión}

Pendiente.


%%%%%%%%%%%%%%%%%%%%%%%%%%%%%%%%%%%%%%%%%%%%%%%%%%%%%%%%%%%%%%%%%%%%%%%%%%%%%%%%
%% Conclusiones                                                               %%
%%%%%%%%%%%%%%%%%%%%%%%%%%%%%%%%%%%%%%%%%%%%%%%%%%%%%%%%%%%%%%%%%%%%%%%%%%%%%%%%


\section{Conclusiones}

Pendiente.


\end{document}
